%%%%%%%%%%%%%%%%%%%%%%%%%%%%%%%%%%%%%%%%%%%%%%%%%%%%%%%%%%%%%%%%%%%%%%
%
% Institut für Rechnergestuetzte Automation
% Forschungsgruppe Industrial Software
% Arbeitsgruppe ESSE
% http://security.inso.tuwien.ac.at/
% lva.security@inso.tuwien.ac.at
% 
%%%%%%%%%%%%%%%%%%%%%%%%%%%%%%%%%%%%%%%%%%%%%%%%%%%%%%%%%%%%%%%%%%%%%%

\documentclass[12pt,a4paper,titlepage,oneside]{scrartcl}
\usepackage{esseProtocol}

%%%%%%%%%%%%%%%%%%%%%%%%%%%%%%%%%%%%%%%%%%%%%%%%%%%%%%%%%%%%%%%%%%%%%%
%
% FOR STUDENTS
%
%%%%%%%%%%%%%%%%%%%%%%%%%%%%%%%%%%%%%%%%%%%%%%%%%%%%%%%%%%%%%%%%%%%%%%

% Group number or "0" for Lab0
\newcommand{\gruppe}{48}
% Date
\newcommand{\datum}{10.5.2013}
% valid values: "Lab0", "Lab1" (be sure to use Uppercase for first character)
\newcommand{\lab}{Lab1}

% name of course, for example: "IT Security in Large IT Infrastructures", "Security for Systems Engineering", "Introduction to Security"
\newcommand{\lvaname}{Security for Systems Engineering}
% number of course, for example: "183.633", "183.637", "183.594"
\newcommand{\lvanr}{183.637}
% year and term, for example: "SS 2012", "WS 2012", "SS 2013", etc.
\newcommand{\semester}{SS 2013}

% Student data in Lab0 or first student of group in Lab1
\newcommand{\studentAName}{Patrick Fleck}
\newcommand{\studentAMatrnr}{1025484}
\newcommand{\studentAEmail}{e1025484@student.tuwien.ac.at}

% second student of group in Lab1, for Lab0 you may leave it unchanged
\newcommand{\studentBName}{Klaus Behrens}
\newcommand{\studentBMatrnr}{0927376}
\newcommand{\studentBEmail}{e0927376@student.tuwien.ac.at}

% second student of group in Lab1, for Lab0 you may leave it unchanged
\newcommand{\studentCName}{Matthias Pernerstorfer}
\newcommand{\studentCMatrnr}{0929329}
\newcommand{\studentCEmail}{e0929329@student.tuwien.ac.at}

% second student of group in Lab1, for Lab0 you may leave it unchanged
\newcommand{\studentDName}{Martin Winklmüller}
\newcommand{\studentDMatrnr}{1025742}
\newcommand{\studentDEmail}{e1025742@student.tuwien.ac.at}

% second student of group in Lab1, for Lab0 you may leave it unchanged
\newcommand{\studentEName}{David Mihola}
\newcommand{\studentEMatrnr}{9902433}
\newcommand{\studentEEmail}{e9902433@student.tuwien.ac.at}

%%%%%%%%%%%%%%%%%%%%%%%%%%%%%%%%%%%%%%%%%%%%%%%%%%%%%%%%%%%%%%%%%%%%%%
%
% DO NOT CHANGE THE FOLLOWING PART
%
%%%%%%%%%%%%%%%%%%%%%%%%%%%%%%%%%%%%%%%%%%%%%%%%%%%%%%%%%%%%%%%%%%%%%%

\newcommand{\dokumenttyp}{Abgabedokument \lab}

\begin{document}

\maketitle
\setcounter{section}{0}
\setcounter{tocdepth}{2}
\tableofcontents

%%%%%%%%%%%%%%%%%%%%%%%%%%%%%%%%%%%%%%%%%%%%%%%%%%%%%%%%%%%%%%%%%%%%%%
%
% CONTENT OF DOCUMENT STARTS HERE
%
%%%%%%%%%%%%%%%%%%%%%%%%%%%%%%%%%%%%%%%%%%%%%%%%%%%%%%%%%%%%%%%%%%%%%%

\section{Lab1a}

\subsection{Hinweise}
\emph{Hinweise:}
\begin{itemize}
    \item Setzen sie alle Variablen nach \emph{FOR STUDENTS} in der .tex File
    \item Ersetzen sie die Platzhalter für ihre Namen und MatNr.
    \item Löschen sie diese Sektion über Hinweise und die folgenden Beispiel-Kapitel
    \item Achten sie auf geforderte Formate und Anforderungen an die Dateinamen
\end{itemize}

\subsection{Zugang zu Tomcat}
Den Zugang zum Tomcat-Server erhält man sehr leicht - zumindest unter Linux. Alle Arbeiten wurden unter Linux
durchgeführt. Für den Zugang ist es notwendig, sich via SSH auf den ESSE-Server zu verbinden, allerdings mit Port Forwarding
auf den angegebenen Tomcat-Server:\linebreak
ssh -p 12345 MATRNR@sela.inso.tuwien.ac.at -L 2222:192.168.20.100:20\linebreak
Dies hat zur Folge, dass der Port 2222 nun auf den Port des Tomcat-Servers geforwarded wird und es somit ein leichtes
ist mittels http://localhost:2222 die Homepage auf dem Server einzusehen.

\subsection{Walter's Private Key}
Nun war es die Aufgabe sich einen SSH-Zugang mit dem User 'walter' zu verschaffen. Hierfür gibt es 2 Möglichkeiten, sich
via SSH zu authentifizieren: 1) Plain-Text Passwort und 2) Private-Key und Public-Key Methode
\linebreak
Natürlich ist es sehr unwahrscheinlich, dass jemand sein Passwort als Plain-Text irgendwo auf einem Server abspeichert. Deshalb
fokusieren wir uns im folgenden auf die 2. Möglichkeit.\linebreak
Unter Linux-Systemen gibt es hierfür einen Ordner .ssh im Home-Verzeichnis des Users. Im Falle von David Mihola würde
dieser unter /home/David/.ssh zu finden sein. Hier liegt nun eine Datei id\_rsa, die im Normalfall den Private Key
des Users, der den SSH-Zugang anfordert beinhaltet.\linebreak
Nun sollte es doch für 'walter' genauso sein. Das heißt wir suchen nach dem Ordner /home/walter/.ssh. Dieser ist aber
natürlich am Server geschützt und kann nicht einfach durch eintippen in Plain-Text in die Adresse des Servers
angesurft werden. Jedoch hat der hier verwendete Tomcat-Server eine Schwachstelle. Mittels Directory-Traversal kann
der Server ausgetricktst werden und der Ordner eingesehen werden. Hier die Adresse der zu untersuchenden Datei:\linebreak
http://localhost:2222/%c0%ae%c0%ae/%c0%ae%c0%ae/%c0%ae%c0%ae/%c0%ae%c0%ae/%c0%ae%c0%ae/home/walter/.ssh/id\_rsa
\linebreak
Directory-Traversal führt hier eine Umwandlung von Plain-Characters in Sonderzeichen ein, wodurch der Server nun die Datei auch
anzeigt. Die Sonderzeichen sind spezielle URI-Zeichenfolgen, die vom Server in Plain-Text umgewandelt werden. Damit kann
man etwaige Schutzmechanismen des Tomcat-Servers umgehen.

\subsection{SSH-Zugang via User 'Walter'}
Wie bekommt man nun einen SSH-Zugang mit einem fremden Private Key?
Wie oben beschrieben, gibt es für jeden User den Ordner .ssh mit den Private Keys für SSH-Zugängen. Es muss nun
möglich sein, sich einen Zugang zum User 'walter' zu verschaffen, indem man sich als dieser ausgibt. Dazu
wird einfach im Ordner /home/David/.ssh eine Datei namens id\_rsa erstellt (falls diese noch nicht existiert). Danach wird
der gefundene Schlüssel von 'walter' hineinkopiert und die Datei gespeichert. Nun ist es nocht notwendig mittels 'chmod
600 id\_rsa' den Zugriffsschutz des Keys auf Permission 600 umzustellen. Andernfalls wird der Server die Verbindung
verweigern, da der Key nicht ausreichend geschützt ist. Das Port-Forwarding über die SSH-Verbindung zum ESSE-Institut
sollte nun noch vorhanden sein. Da wir ja über den Port 2222 direkt auf den Tomcat-Server zugreifen können, ist es nun
möglich sich mittels 'ssh -p 2222 walter@localhost' mit dem Benutzer 'walter' einzuloggen. Dies funktioniert aus folgendem
Grund:\linebreak
Der Server speichert (im .ssh-Ordner) zusätzlich noch den Public-Key jedes Users. Wird nun eine Verbindung angefordert,
so muss der User, der dies durchführt, einen Private-Key in seinem .ssh-Ordner besitzen, der zu einem Public-Key am Server passt.
Wir haben uns den Private-Key von 'walter' in unseren eigenen .ssh-Ordner kopiert und der Server glaubt nun, wir seien
auch dieser User. 

\section{Ueberschrift 2}

\subsection{Sub-Ueberschrift 1}
Lorem ipsum dolor sit amet, consetetur sadipscing elitr, sed diam nonumy eirmod tempor invidunt ut labore et dolore magna aliquyam erat, sed diam voluptua. 

\subsection{Sub-Ueberschrift 2}
Lorem ipsum dolor sit amet, consetetur sadipscing elitr, sed diam nonumy eirmod tempor invidunt ut labore et dolore magna aliquyam erat, sed diam voluptua. At vero eos et accusam et justo duo dolores et ea rebum. 

\subsection{Sub-Ueberschrift 3}
Lorem ipsum dolor sit amet, consetetur sadipscing elitr, sed diam nonumy eirmod tempor invidunt ut labore et dolore magna aliquyam erat, sed diam voluptua. 

\section{Beispiele}

\subsection{Source Code formatieren}
Es folgen einige Beispiele wie Sourcecode in diesem Dokument formatiert und referenziert werden kann
(\hyperref[code:beispiel1]{siehe Listing~\ref*{code:beispiel1} auf Seite~\pageref*{code:beispiel1}} und \hyperref[code:beispiel2]{siehe Listing~\ref*{code:beispiel2} auf Seite~\pageref*{code:beispiel2}}).

Ebenso können kurzer Code oder kurze Befehle direkt in der Zeile in einem \lstinline{lstinline Block} mit typengleicher Schrift formatiert werden.

\lstinputlisting[caption=Example C/C++ file,label=code:beispiel1,style=c]{example.c}

\begin{lstlisting}[caption=Example bash script,label=code:beispiel2,style=simple]
#!/bin/bash
echo "Bash version ${BASH_VERSION}..."
for i in {0..10..2}
  do
     echo "Welcome $i times"
 done

echo "some very very very very very very very very very very very very very very very very very very very very long string"

exit 0;
\end{lstlisting}

\subsection{Bilder}

Es folgen einige Beispiele wie Bilder in diesem Dokument eingefuegt werden koennen
(\hyperref[fig:logo1]{siehe Abbildung~\ref*{fig:logo1} auf Seite~\pageref*{fig:logo1}}).

\begin{figure}[h!]
  \centering
  \fbox{
    \includegraphics[width=0.4\textwidth]{./imgs/esse-logo-bw.png}
  }
  \caption{ESSE Logo}
  \label{fig:logo1}
\end{figure}


\end{document}


