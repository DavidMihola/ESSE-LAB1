\section{App1 - Blog in Java mit JSP}

\subsection{Vulnerability: Cross-Site Scripting}

Das Blog ist anfällig für Cross-Site Scripting. Das Problem liegt darin, dass die Eingabedaten beim Erzeugen eines neuen Blog-Eintrags nur unzureichend gefiltert werden. Beim erneuten Anzeigen eines betroffenen Blog-Eintrags werden diese - potentiell gefährlichen - Daten dann an den Browsers gesendet und dort ausgeführt. 

Ein konkretes Beispiel für einen Angriff: Der Angreifer "evil\_häcker" legt einen neuen Blog-Eintrag an. Im Formular auf newPost.html gibt er als Benutzernamen - wahrheitsgemäß - "evil\_häcker" ein. Als Text des Blog-Eintrags wählt er (MUSS ICH NOCH EINFÜGEN).

Im Servlet, das den Post-Request entgegen nimmt (gui.controller.BlogServlet) wird die Eingabe zwar validiert, bevor sie in die (In-Memory-)Datenbank geschrieben wird. Leider ist der verwendete Validator (gui.helper.ScriptTagValidator) nicht ausreichend. Die verwendete Regulary Expression matcht nur Script-Tags, die keine Attribute zwischen dem Wort "script" und der schließenden Spitz-Klammer enthalten. Durch die Angabe "src="NOCH EINFÜGEN"" lässt sich der Validator also in die Irre führen und lässt die Eingabe ungefiltert durch.

Da bei der Ausgabe der Blog-Einträge die Inhalte überhaupt nicht mehr gefiltert werden, wird also der folgende Text als Blog-Eintrag wieder ausgegeben (z. B. in archive.jsp). Im Browser wird das Script-Tag dann so behandelt, als würde es aus vertrauenswürdiger Quelle kommen: Das Script von der angegebenen URL wird heruntergeladenen und ausgeführt. Im konkreten Fall wird dadurch an der Stelle, an der sich das Script-Tag befand der Text "This is the evil script" ausgegeben.

Der Vollständigkeit halber soll darauf hinwiesen werden, wenn der Angreifer das Script-Tag als Autoren-Namen und nicht als Blog-Text eingegeben hätte, da diese beiden Werte auf exakt dieselbe Art behandelt werden.

\subsection{Lösung: Escapen beim erneuten Anzeigen}

Die beste Lösung, um Cross-Site Scripting in diesem konkreten Fall vorzubeugen, liegt darin, die Blog-Daten (Autoren und Inhalte) bei der erneuten Ausgabe zu filtern. Dabei werden potentiell gefährliche Zeichen wie < und > durch entsprechende Escape-Sequenzen ersetzt, der Angriffs-Text aus dem Beispiel wird dadurch zu: ...

Im Browser werden diese Escape-Sequenzen dann durch die entsprechenden Zeichen ersetzt und am Bildschirm ausgegeben. Im Unterschied zum nicht gefilterten Text wird das Script-Tag allerdings \emph{nicht ausgeführt} - der Browser behandelt es als ganz normalen Text.

Eine kurze Recherche im Internet zeigt, dass Varialben-Werte in Servlet-Tags in JSP generell nicht gefiltert/escaped werden. Im Gegensatz dazu werden Variablen-Werte in Expression Language-Tags in JSP (per Default) \emph{immer} escaped. Um die Ausgabe-Werte zu escapen ist es also lediglich notwendig, die entsprechenden Servlet-Abschnitte durch Expression Language zu ersetzen, genauer gesagt durch c:forEach für den Schleifen-Durchgang (ist das überhaupt notwendig?) und durch c:out für die eigentliche Ausgabe. Das Tag c:out führt dann das eigentlich Escapen durch.

Die beiden folgenden Code-Beispiele zeigen die Änderungen am Beispiel von archive.jsp. Das erste Beispiel zeigt die anfällige Variante, das zweite Beispiel die verbesserte Version.

EINFÜGEN

Man beachte, dass zwei weitere Veränderungen notwendig sind, um die EL-Tags verwenden zu können.
\begin{enumerate}
\item In der betreffenden JSP-Datei muss das Tag-Libary durch eine entsprechende taglib-Direktive deklariert werden. 
\item Da Tomcat nicht mit den Archiven der Java Standard Tag Library ausgeliefert wird, müssen diese heruntergeladen und im Verzeichnis WEB-INF/lib abgelegt werden.
\end{enumerate}

